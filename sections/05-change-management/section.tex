\section{Change Management and Version Control}

\subsection{Principle of Scope Preservation}

Models are approved for use only within their declared Operating Context and validated configuration.

Any change that alters the data, assumptions, objectives, constraints, or usage of a model alters the question the model is answering and therefore constitutes a \textbf{change in model scope}.

Changes in scope must be governed explicitly. Undocumented or implicit changes are prohibited.

\subsection{Definition of a Model Change}

The following changes constitute a \textbf{new model version} and are subject to this section:

\subsubsection{Data and Feature Changes}

\begin{itemize}
  \item addition, removal, or modification of data sources;
  \item changes in data frequency, aggregation, or preprocessing;
  \item changes to feature definitions, transformations, or lags.
\end{itemize}

\subsubsection{Objective and Estimation Changes}

\begin{itemize}
  \item changes to training objectives or loss functions;
  \item changes to estimation methods or fitting procedures;
  \item changes to regularization, constraints, or optimization targets.
\end{itemize}

\subsubsection{Portfolio and Execution Changes}

\begin{itemize}
  \item changes to portfolio construction rules;
  \item changes to rebalancing frequency or timing;
  \item changes to execution logic or order assumptions.
\end{itemize}

\subsubsection{Assumption and Constraint Changes}

\begin{itemize}
  \item changes to transaction cost or market impact models;
  \item changes to liquidity or capacity assumptions;
  \item changes to risk constraints or exposure limits.
\end{itemize}

\subsubsection{Usage and Decision Changes}

\begin{itemize}
  \item changes to the decisions the model informs;
  \item changes in capital allocation linked to the model;
  \item changes in integration with other models or systems.
\end{itemize}

Cosmetic, stylistic, or documentation-only changes do not constitute a model change but must still be logged.

\subsection{Model Versioning Requirements}

Each model version must be uniquely identified and recorded in the Model Registry.

For each version, the following must be retained:

\begin{itemize}
  \item version identifier and timestamp;
  \item description of changes relative to the prior version;
  \item rationale for the change;
  \item updated Operating Context, if applicable;
  \item summary comparison of behavior relative to the prior version.
\end{itemize}

Prior versions may not be overwritten or deleted.

\subsection{Classification of Changes}

Model changes are classified based on their impact.

\subsubsection{Non-Impacting Changes}

\begin{itemize}
  \item Documentation clarifications
  \item Code refactoring with no behavioral impact
\end{itemize}

\textbf{Governance Requirement}\\
Logging only. No re-validation required.

\subsubsection{Diagnostic-Scope Changes}

\begin{itemize}
  \item Changes affecting exploratory or diagnostic use only
  \item No effect on implementability or capital allocation
\end{itemize}

\textbf{Governance Requirement}\\
Research owner signoff and documentation update.

\subsubsection{Translational or Deployable Changes}

\begin{itemize}
  \item Any change affecting implementation behavior, risk, or performance
  \item Any change affecting live or potential capital allocation
\end{itemize}

\textbf{Governance Requirement}\\
Full re-validation and approval prior to use.

\subsection{Prohibition on Retroactive Justification}

Model changes may not be justified using performance observed prior to the change.

Evaluation of a new model version must be conducted prospectively relative to its declared Operating Context.

Backfilling or reinterpretation of historical results under modified assumptions is prohibited.

\subsection{Approval Authority for Changes}

Approval authority depends on model classification and change impact:

\begin{itemize}
  \item \textbf{Diagnostic models}: research owner approval
  \item \textbf{Translational models}: research and risk management approval
  \item \textbf{Deployable models}: investment committee or delegated authority approval
\end{itemize}

Approval must be documented prior to use.

\subsection{Emergency Changes}

In exceptional circumstances (e.g., market structure changes, operational issues), temporary changes may be implemented to mitigate risk.

Emergency changes must:

\begin{itemize}
  \item be clearly labeled as temporary;
  \item be documented within one business day; and
  \item undergo formal review and approval as soon as practicable.
\end{itemize}

Emergency status does not exempt the model from subsequent validation.

\subsection{Audit Trail and Traceability}

The firm must maintain an auditable trail linking:

\begin{itemize}
  \item model versions;
  \item operating contexts;
  \item validation results;
  \item approvals; and
  \item deployment dates.
\end{itemize}

The audit trail must support reconstruction of:

\begin{itemize}
  \item what model version was used;
  \item under what assumptions; and
  \item for what decisions at any point in time.
\end{itemize}

\subsection{Consequences of Unauthorized Changes}

Unauthorized or undocumented model changes constitute a governance breach.

Consequences may include:

\begin{itemize}
  \item immediate suspension of model use;
  \item reduction or removal of capital allocation;
  \item mandatory re-validation; and
  \item escalation to senior management or the Investment Committee.
\end{itemize}

\subsection{Guiding Rule}

\begin{quote}
\textbf{A model that has changed is a different model. Governance exists to ensure that this is recognized before capital is affected.}
\end{quote}
