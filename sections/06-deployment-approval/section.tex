\section{Deployment Approval and Capital Allocation Controls}

\subsection{Principle of Controlled Deployment}

No quantitative model may influence live trading decisions or capital allocation unless it has been formally approved for deployment under this framework.

Deployment approval is \textbf{context-specific}. Approval to deploy a model under one Operating Context does not authorize its use under materially different conditions.

The burden of proof for deployment rests with the model owner.

\subsection{Preconditions for Deployment Approval}

A model may be considered for deployment only if all of the following conditions are met:

\subsubsection{Classification Eligibility}

\begin{itemize}
  \item The model is classified as a \textbf{Deployable Model} in the Model Registry.
  \item Any prior use as a diagnostic or translational model has been explicitly closed or re-scoped.
\end{itemize}

\subsubsection{Completed Validation}

\begin{itemize}
  \item Validation has been conducted in accordance with Section 4.
  \item Evaluation reflects the declared Operating Context.
  \item Trade-offs, sensitivities, and limitations are documented.
\end{itemize}

\subsubsection{Documented Operating Context}

\begin{itemize}
  \item The Operating Context is complete, current, and approved.
  \item Scope boundaries and exclusions are explicit.
  \item Known failure modes are documented.
\end{itemize}

\subsubsection{Independent Review}

\begin{itemize}
  \item Risk management or an independent governance function has reviewed the model.
  \item Material review findings have been resolved or formally accepted.
\end{itemize}

\subsection{Deployment Approval Materials}

Deployment requests must include, at a minimum:

\begin{itemize}
  \item a summary of intended use and decision impact;
  \item the declared Operating Context;
  \item validation results and trade-off analysis;
  \item identified risks, fragilities, and failure modes;
  \item proposed capital limits and scaling plan;
  \item monitoring metrics and escalation triggers.
\end{itemize}

Approval materials must distinguish clearly between:

\begin{itemize}
  \item evidence generated under approved assumptions; and
  \item exploratory or out-of-context analyses.
\end{itemize}

\subsection{Approval Authority}

Deployment approval authority depends on the model's impact and capital at risk:

\begin{itemize}
  \item \textbf{Low-impact deployable models}: delegated authority (as defined by firm policy).
  \item \textbf{Material-impact deployable models}: Investment Committee approval.
  \item \textbf{Firm-wide or systemic models}: senior management or board-level oversight.
\end{itemize}

Approval authority must be documented in the Model Registry.

\subsection{Capital Allocation Controls}

\subsubsection{Initial Capital Limits}

All deployed models must begin with conservative capital limits.

Initial limits must reflect:

\begin{itemize}
  \item uncertainty in model behavior;
  \item sensitivity to assumptions;
  \item liquidity and operational considerations.
\end{itemize}

Capital limits may not be increased automatically based on short-term performance.

\subsubsection{Scaling Conditions}

Any increase in capital allocation must be conditioned on:

\begin{itemize}
  \item sustained behavior within expected ranges;
  \item stability of key monitoring metrics;
  \item absence of Operating Context violations.
\end{itemize}

Scaling decisions require review and approval proportional to the increase in exposure.

\subsubsection{Prohibited Allocation Practices}

The following practices are prohibited:

\begin{itemize}
  \item allocating capital based solely on backtested performance;
  \item extrapolating performance beyond evaluated scale;
  \item inferring robustness from short-lived success.
\end{itemize}

\subsection{Deployment Constraints and Controls}

Deployed models must operate within predefined constraints, including:

\begin{itemize}
  \item maximum turnover and trading intensity;
  \item exposure and concentration limits;
  \item drawdown or loss thresholds;
  \item operational and execution constraints.
\end{itemize}

Constraints must be enforceable via systems controls where feasible.

\subsection{Conditional and Limited Deployment}

Where appropriate, models may be approved for \textbf{conditional deployment}, including:

\begin{itemize}
  \item paper trading with live monitoring;
  \item shadow capital allocations;
  \item partial or capped integration into existing strategies.
\end{itemize}

Conditional deployment must include:

\begin{itemize}
  \item explicit success and failure criteria; and
  \item a defined review timeline.
\end{itemize}

\subsection{Suspension and De-Escalation Authority}

Risk management and designated governance functions have the authority to:

\begin{itemize}
  \item reduce capital allocation;
  \item restrict model usage; or
  \item suspend deployment
\end{itemize}

in response to:

\begin{itemize}
  \item Operating Context violations;
  \item monitoring breaches;
  \item unexplained behavioral changes; or
  \item external events invalidating assumptions.
\end{itemize}

Suspension actions must be documented and reviewed.

\subsection{Post-Approval Obligations}

Approval for deployment carries ongoing obligations:

\begin{itemize}
  \item compliance with monitoring requirements (Section 7);
  \item prompt disclosure of assumption breaches;
  \item timely initiation of reviews when conditions change.
\end{itemize}

Failure to meet post-approval obligations may result in withdrawal of approval.

\subsection{Governing Standard}

\begin{quote}
\textbf{Deployment approval reflects confidence in how a model behaves under defined conditions, not belief in its ability to perform under all conditions.}
\end{quote}
