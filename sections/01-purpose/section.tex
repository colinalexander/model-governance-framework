\section{Purpose and Governance Objectives}

\subsection{Purpose of the Model Governance Framework}

This framework establishes firm-wide standards for the design, evaluation, approval, deployment, monitoring, and retirement of quantitative models used in investment-related activities.

The objective of the framework is to ensure that:

\begin{itemize}
  \item models are used only for decisions they are demonstrably fit to support;
  \item conclusions drawn from models are valid under clearly specified operating conditions;
  \item changes in assumptions, data, or usage are identified and governed as changes in scope; and
  \item model-related risks are identified, documented, and managed proactively.
\end{itemize}

This framework is intended to protect the firm from losses arising not only from model error, but from \textbf{misinterpretation, misuse, or over-extension of otherwise sound models}.

\subsection{Definition of a Model}

For the purposes of this framework, a \emph{model} is defined as any quantitative, statistical, algorithmic, or rules-based system that:

\begin{itemize}
  \item transforms data into signals, forecasts, rankings, or decisions; and
  \item is referenced, directly or indirectly, in forming investment views, portfolio allocations, trade decisions, risk assessments, or execution strategies.
\end{itemize}

This definition includes, but is not limited to:

\begin{itemize}
  \item predictive and classification models;
  \item ranking and scoring systems;
  \item portfolio construction and optimization routines;
  \item risk, exposure, and scenario models;
  \item execution and cost models; and
  \item externally sourced or third-party models adopted by the firm.
\end{itemize}

Ad hoc analyses or exploratory tools become subject to this framework once they are cited in decision-making, approval materials, or capital allocation discussions.

\subsection{Governance Objectives}

The framework is designed to achieve the following governance objectives:

\subsubsection{Contextual Integrity}

Ensure that each model is evaluated, approved, and used only within the operating conditions under which its behavior has been examined.

\subsubsection{Scope Control}

Prevent implicit expansion of model use beyond its validated purpose through clear definition of intended use and formal control of changes.

\subsubsection{Evidence Discipline}

Ensure that claims made about a model's performance, robustness, or suitability do not exceed the evidence generated under its declared operating conditions.

\subsubsection{Risk Proportionality}

Apply governance controls in proportion to the model's influence on investment decisions and the amount of capital or risk it affects.

\subsubsection{Transparency and Accountability}

Maintain a clear record of model assumptions, limitations, versions, and approvals to support internal review, audit, and learning.

\subsection{Guiding Principles}

All subsequent sections of this framework are governed by the following principles.

\subsubsection{Validity Is Conditional}

Model outputs are valid only within the conditions under which the model was evaluated. Performance observed under one set of assumptions does not justify use under materially different conditions.

\subsubsection{Changes Are Substantive, Not Cosmetic}

Changes to data, assumptions, horizons, objectives, or constraints alter the question being answered by a model and therefore constitute changes in model scope.

\subsubsection{Trade-Offs Are Fundamental}

Model quality cannot be reduced to a single metric. All models embody trade-offs (e.g., performance vs. turnover, responsiveness vs. stability), which must be made explicit.

\subsubsection{Fragility Is a Risk Signal}

Sensitivity of model behavior to reasonable changes in assumptions or conditions is treated as a model risk characteristic and must be disclosed and managed.

\subsubsection{Governance Follows Impact}

The rigor of validation, approval, and monitoring increases with the potential impact of the model on capital allocation, risk exposure, and firm outcomes.

\subsection{Relationship to Other Firm Policies}

This framework operates in conjunction with, and does not replace:

\begin{itemize}
  \item investment approval and capital allocation policies;
  \item risk management and limits frameworks;
  \item compliance and market conduct policies; and
  \item technology and operational controls.
\end{itemize}

In the event of conflict, the most restrictive applicable requirement governs.

\subsection{Policy Authority and Enforcement}

Compliance with this framework is mandatory.

\begin{itemize}
  \item Use of a model outside its approved scope constitutes a governance breach.
  \item Failure to document assumptions or changes as required may result in suspension of model use.
  \item Persistent or material violations may lead to escalation to senior management and the Investment Committee.
\end{itemize}

This framework is reviewed periodically and updated as necessary to reflect changes in firm strategy, market structure, or regulatory expectations.
