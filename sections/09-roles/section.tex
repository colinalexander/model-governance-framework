\section{Roles, Responsibilities, and Accountability}

\subsection{Purpose of Role Definition}

Effective model governance requires clear ownership, independent challenge, and defined decision authority.

This section assigns responsibilities to specific functions to ensure that:

\begin{itemize}
  \item model assumptions are owned and maintained;
  \item governance decisions are independent of model development;
  \item accountability is preserved across the model lifecycle; and
  \item no model remains active without a responsible owner.
\end{itemize}

\subsection{Model Owner}

Each model must have a designated \textbf{Model Owner}, recorded in the Model Registry.

\subsubsection{Responsibilities of the Model Owner}

The Model Owner is responsible for:

\begin{itemize}
  \item ensuring accuracy and completeness of model documentation;
  \item declaring and maintaining the Operating Context;
  \item disclosing assumptions, limitations, and known fragilities;
  \item initiating validation, review, or re-approval when conditions change;
  \item responding to monitoring breaches and review findings; and
  \item recommending restriction or retirement when appropriate.
\end{itemize}

The Model Owner must have sufficient technical understanding of the model to explain its behavior and limitations.

\subsection{Governance Reviewer (Independent Oversight)}

Each deployable and translational model must have an assigned \textbf{Governance Reviewer}, independent of model development.

\subsubsection{Responsibilities of the Governance Reviewer}

The Governance Reviewer is responsible for:

\begin{itemize}
  \item independent challenge of model assumptions and scope;
  \item assessing alignment between evidence and claims;
  \item reviewing validation design and results;
  \item enforcing classification and Operating Context discipline; and
  \item recommending approval, restriction, or escalation actions.
\end{itemize}

The Governance Reviewer has authority to withhold approval pending resolution of material concerns.

\subsection{Risk Management Function}

The Risk Management function has primary responsibility for:

\begin{itemize}
  \item defining validation and monitoring standards for deployable models;
  \item conducting or overseeing independent model reviews;
  \item monitoring live model behavior and context adherence;
  \item initiating escalation, restriction, or suspension actions; and
  \item reporting material model risks to senior management and the Investment Committee.
\end{itemize}

Risk Management authority to restrict or suspend model use is independent of research or portfolio management.

\subsection{Investment Committee}

The Investment Committee has ultimate authority over:

\begin{itemize}
  \item approval of deployable models with material capital impact;
  \item determination of capital allocation limits and scaling conditions;
  \item acceptance of residual model risks; and
  \item resolution of escalated governance issues.
\end{itemize}

The Investment Committee must ensure that deployment decisions are supported by evidence consistent with the declared Operating Context.

\subsection{Research Function}

The Research function is responsible for:

\begin{itemize}
  \item developing and testing models in accordance with this framework;
  \item producing evidence aligned with intended use;
  \item documenting trade-offs, sensitivities, and limitations; and
  \item engaging constructively with independent review and challenge.
\end{itemize}

Research is accountable for methodological rigor, not for deployment outcomes.

\subsection{Operations and Technology}

Operations and Technology functions are responsible for:

\begin{itemize}
  \item implementing system controls that enforce approved model usage;
  \item ensuring operational feasibility and reliability;
  \item supporting monitoring, logging, and audit requirements; and
  \item flagging operational issues that may affect model behavior.
\end{itemize}

Operational constraints are treated as binding inputs to model governance.

\subsection{Compliance and Legal (Where Applicable)}

Compliance and Legal functions support this framework by:

\begin{itemize}
  \item advising on regulatory or contractual obligations;
  \item ensuring adherence to applicable laws and policies; and
  \item reviewing documentation and processes for audit readiness.
\end{itemize}

Compliance does not substitute for governance but reinforces it.

\subsection{Incentive Alignment and Cultural Safeguards}

The framework recognizes that incentives and time pressure can distort judgment and encourage premature deployment or scope creep.

Accordingly:

\begin{itemize}
  \item Governance exists to counteract short-term performance pressure and protect long-term capital integrity.
  \item Attempts to bypass, delay, or minimize governance requirements are treated as \textbf{risk signals} and must be escalated.
  \item Incentive conflicts should be disclosed and managed as part of review and approval processes.
\end{itemize}

\subsection{Accountability and Escalation}

Accountability for model-related decisions follows declared roles.

\begin{itemize}
  \item Decisions to deploy, restrict, or retire models must be attributable to defined authorities.
  \item Failure to fulfill assigned responsibilities may result in restriction of model use or escalation to senior management.
  \item Disagreements between functions are resolved by escalation, not informal override.
\end{itemize}

\subsection{Governing Principle}

\begin{quote}
\textbf{Clear ownership enables clear judgment. Governance fails when responsibility is diffuse or implicit.}
\end{quote}
