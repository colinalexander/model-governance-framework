\section{Declared Operating Context}

\subsection{Requirement for a Declared Operating Context}

Every model subject to this framework must include a formally declared \textbf{Operating Context}.

The Operating Context defines the specific conditions under which the model has been evaluated and approved for use. Model outputs are considered valid \textbf{only} within this declared context.

Models may not be used, referenced, or interpreted outside their declared Operating Context without formal re-evaluation and approval.

\subsection{Components of the Operating Context}

The Operating Context must be documented in a structured, reviewable format and must include, at a minimum, the following elements.

\subsubsection{Asset Universe}

\begin{itemize}
  \item Instruments, asset classes, or markets covered
  \item Inclusion and exclusion criteria
  \item Treatment of corporate actions, listings, and delistings
\end{itemize}

\subsubsection{Data Inputs}

\begin{itemize}
  \item Data sources and vendors
  \item Data frequency and aggregation
  \item Lookback windows and lags
  \item Preprocessing, filtering, and normalization procedures
\end{itemize}

\subsubsection{Decision Horizon}

\begin{itemize}
  \item Prediction or decision horizon
  \item Relationship between signal horizon and trading horizon
  \item Alignment with portfolio or execution timing
\end{itemize}

\subsubsection{Portfolio Construction and Execution}

\begin{itemize}
  \item Position sizing methodology
  \item Portfolio constraints (e.g., leverage, gross and net exposure)
  \item Rebalancing frequency and rules
  \item Execution assumptions (e.g., order types, participation rates)
\end{itemize}

\subsubsection{Transaction Costs and Market Impact}

\begin{itemize}
  \item Cost model used in evaluation
  \item Assumptions regarding spreads, fees, and slippage
  \item Treatment of market impact and liquidity limits
\end{itemize}

\subsubsection{Risk Constraints}

\begin{itemize}
  \item Exposure limits (factor, sector, asset-level)
  \item Risk metrics used (e.g., volatility, drawdown)
  \item Integration with firm risk policies
\end{itemize}

\subsubsection{Capacity and Scale Assumptions}

\begin{itemize}
  \item Capital levels evaluated
  \item Scaling behavior assumptions
  \item Known capacity limits or degradation effects
\end{itemize}

\subsubsection{Intended Decision Supported}

\begin{itemize}
  \item Specific decisions the model is intended to inform
  \item Decisions the model is explicitly \textbf{not} designed to support
\end{itemize}

\subsection{Scope Boundaries and Exclusions}

Each Operating Context must explicitly state:

\begin{itemize}
  \item conditions under which model performance is expected to degrade;
  \item assumptions that, if violated, invalidate conclusions; and
  \item market environments or regimes for which the model has not been evaluated.
\end{itemize}

Implicit assumptions are not permitted. Undeclared assumptions are treated as \textbf{unapproved scope expansion}.

\subsection{Context Consistency and Evidence Alignment}

All validation, performance evaluation, and monitoring metrics must be consistent with the declared Operating Context.

Evidence generated under different assumptions---such as alternative cost models, rebalancing frequencies, or horizons---must be clearly labeled as \textbf{out-of-context} and may not be used to justify in-context deployment.

Where multiple contexts are evaluated, results must be presented separately and not aggregated without explicit justification.

\subsection{Context Changes and Re-Approval}

\subsubsection{Definition of a Context Change}

Any change to the components listed in Section 3.2 constitutes a change in Operating Context.

Examples include, but are not limited to:

\begin{itemize}
  \item expanding or altering the asset universe;
  \item changing data sources or frequency;
  \item modifying portfolio construction rules;
  \item updating transaction cost or liquidity assumptions;
  \item altering rebalancing cadence or decision horizon.
\end{itemize}

\subsubsection{Governance Requirements for Context Changes}

\begin{itemize}
  \item Context changes affecting diagnostic use require documentation update.
  \item Context changes affecting translational or deployable use require re-validation and approval prior to use.
\end{itemize}

Context changes may not be implemented retroactively.

\subsection{Context Hierarchy and Conservative Interpretation}

When uncertainty exists regarding applicable context, the model shall be governed under the \textbf{most restrictive interpretation}.

Evidence generated under relaxed assumptions does not supersede evidence generated under stricter assumptions unless explicitly approved.

\subsection{Enforcement and Violations}

Use of a model outside its declared Operating Context constitutes a governance violation.

Violations may result in:

\begin{itemize}
  \item immediate restriction or suspension of model use;
  \item reduction of capital exposure;
  \item mandatory review by risk management; and
  \item escalation to senior management or the Investment Committee, as appropriate.
\end{itemize}

\subsection{Operating Context Documentation}

The declared Operating Context must be:

\begin{itemize}
  \item stored alongside the model in the Model Registry;
  \item versioned and auditable;
  \item referenced in all approval and review materials; and
  \item reviewed whenever material market, operational, or strategic changes occur.
\end{itemize}
