\section{Review, Escalation, and Model Retirement}

\subsection{Purpose of Review and Escalation}

This section defines the processes by which models are:

\begin{itemize}
  \item reviewed in light of new information;
  \item escalated when risks emerge; and
  \item restricted or retired when continued use is no longer justified.
\end{itemize}

The objective is to ensure that models are \textbf{reassessed deliberately}, rather than implicitly abandoned or defended through inertia.

\subsection{Periodic Model Review}

\subsubsection{Review Frequency}

All deployable models must undergo a formal review at least quarterly.

Additional reviews are required when:

\begin{itemize}
  \item material market or structural changes occur;
  \item Operating Context assumptions are challenged;
  \item monitoring thresholds are breached; or
  \item capital allocation is materially increased.
\end{itemize}

\subsubsection{Review Scope}

Periodic reviews must assess:

\begin{itemize}
  \item continued validity of the declared Operating Context;
  \item relevance of assumptions and constraints;
  \item consistency of observed behavior with validated expectations;
  \item adequacy of monitoring metrics and thresholds; and
  \item appropriateness of current model classification and capital allocation.
\end{itemize}

Reviews must be documented and logged in the Model Registry.

\subsection{Escalation Triggers}

Escalation is required when one or more of the following occur:

\begin{itemize}
  \item repeated or unexplained monitoring breaches;
  \item persistent divergence between expected and observed behavior;
  \item assumption violations that materially affect model validity;
  \item operational or execution failures affecting model performance;
  \item external events that invalidate key elements of the Operating Context.
\end{itemize}

Escalation may be initiated by research, risk management, operations, or senior management.

\subsection{Escalation Process}

\subsubsection{Initial Assessment}

Upon escalation:

\begin{itemize}
  \item risk management conducts an initial assessment of severity and scope;
  \item interim controls (e.g., capital reduction) may be applied; and
  \item the model may be placed in \textbf{Restricted} status.
\end{itemize}

\subsubsection{Formal Review}

If escalation persists or is material:

\begin{itemize}
  \item a formal review is conducted involving research, risk, and relevant stakeholders;
  \item findings are documented, including root-cause analysis where appropriate; and
  \item recommendations are made regarding remediation, restriction, or retirement.
\end{itemize}

\subsection{Model Restriction}

A model may be restricted when:

\begin{itemize}
  \item risk is elevated but potentially remediable; or
  \item further evidence is required to restore confidence.
\end{itemize}

Restrictions may include:

\begin{itemize}
  \item reduced capital allocation;
  \item narrowed scope of use;
  \item increased monitoring intensity; or
  \item temporary suspension of deployment.
\end{itemize}

Restriction status and conditions for lifting restrictions must be documented.

\subsection{Dormant Models and Archiving}

Models that are no longer used to support decisions may not remain in active status without clear intent.

Accordingly:

\begin{itemize}
  \item A model that is no longer referenced in decisions must be moved to \textbf{Research} status (if still under development) or \textbf{Retired} status (if no active development is planned).
  \item Dormant models must be clearly labeled in the Model Registry and are not eligible for decision use until re-validated and approved.
  \item Archiving is for audit and learning purposes only; archived models do not retain deployment rights.
\end{itemize}

\subsection{Model Retirement}

\subsubsection{Retirement Criteria}

A model must be retired when:

\begin{itemize}
  \item its Operating Context assumptions no longer hold;
  \item performance or behavior degrades persistently outside acceptable bounds;
  \item remediation is impractical or ineffective; or
  \item strategic, operational, or regulatory considerations require withdrawal.
\end{itemize}

Retirement is a governance action and does not imply fault or error.

\subsubsection{Retirement Process}

Upon retirement:

\begin{itemize}
  \item the model is removed from live systems;
  \item capital allocation is fully withdrawn;
  \item the model status is updated to \textbf{Retired} in the Model Registry; and
  \item final documentation is completed, including reasons for retirement.
\end{itemize}

\subsection{Post-Retirement Review}

Where appropriate, a post-retirement review is conducted to:

\begin{itemize}
  \item document lessons learned;
  \item identify governance or validation improvements; and
  \item inform future model development.
\end{itemize}

Post-retirement reviews focus on process and assumptions, not individual attribution.

\subsection{Re-Activation of Retired Models}

Retired models may not be re-activated without:

\begin{itemize}
  \item re-classification;
  \item re-validation under a current Operating Context; and
  \item formal re-approval.
\end{itemize}

Prior approval does not carry forward automatically.

\subsection{Documentation and Auditability}

All review, escalation, restriction, and retirement actions must be:

\begin{itemize}
  \item documented;
  \item time-stamped;
  \item attributable to responsible parties; and
  \item retained for audit and learning purposes.
\end{itemize}

\subsection{Governing Principle}

\begin{quote}
\textbf{Models should exit as deliberately as they enter. Governance exists to ensure that retirement is timely, reasoned, and unambiguous.}
\end{quote}
