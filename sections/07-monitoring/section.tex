\section{Ongoing Monitoring and Risk Controls}

\subsection{Purpose of Ongoing Monitoring}

Ongoing monitoring is required to ensure that deployed models continue to behave in a manner consistent with their declared Operating Context and validated expectations.

Monitoring is not intended to evaluate whether a model is ``working,'' but to detect:

\begin{itemize}
  \item deviations from expected behavior;
  \item violations of underlying assumptions; and
  \item emerging risks that may not be visible through performance alone.
\end{itemize}

Monitoring obligations apply to all deployable models for the duration of their use.

\subsection{Monitoring Focus}

Monitoring must emphasize \textbf{model behavior and context adherence}, not solely realized P\&L.

At a minimum, the following dimensions must be monitored, where applicable.

\subsubsection{Implementation Behavior}

\begin{itemize}
  \item turnover and trading intensity relative to expected ranges;
  \item transaction costs and slippage relative to assumptions;
  \item execution quality and operational performance.
\end{itemize}

\subsubsection{Signal and Decision Stability}

\begin{itemize}
  \item stability of model outputs or rankings;
  \item frequency and magnitude of decision changes;
  \item divergence from historical behavior patterns.
\end{itemize}

\subsubsection{Risk and Exposure Profile}

\begin{itemize}
  \item factor, sector, and asset-level exposures;
  \item concentration metrics;
  \item utilization of risk limits.
\end{itemize}

\subsubsection{Sensitivity to Operating Conditions}

\begin{itemize}
  \item sensitivity to costs, liquidity, or volatility changes;
  \item degradation under stressed or adverse conditions;
  \item scale effects as capital changes.
\end{itemize}

\subsection{Monitoring Metrics and Thresholds}

For each deployable model, a set of \textbf{monitoring metrics and thresholds} must be defined at deployment.

Thresholds should reflect:

\begin{itemize}
  \item validated operating ranges;
  \item known fragilities; and
  \item risk tolerance appropriate to the model's impact.
\end{itemize}

Thresholds must be:

\begin{itemize}
  \item documented;
  \item reviewable; and
  \item enforceable through systems controls where feasible.
\end{itemize}

\subsection{Detection of Operating Context Violations}

Monitoring must explicitly identify \textbf{Operating Context violations}, including but not limited to:

\begin{itemize}
  \item trading behavior inconsistent with declared assumptions;
  \item turnover or cost levels materially exceeding evaluated ranges;
  \item exposure patterns outside approved limits;
  \item use of the model for decisions beyond its intended scope.
\end{itemize}

Detection of an Operating Context violation requires immediate review.

\subsection{Response to Monitoring Breaches}

\subsubsection{Initial Response}

Upon breach of a monitoring threshold:

\begin{itemize}
  \item the model owner must be notified promptly;
  \item risk management must assess severity and persistence; and
  \item interim controls (e.g., capital reduction) may be applied.
\end{itemize}

\subsubsection{Escalation}

If breaches are material, persistent, or unexplained:

\begin{itemize}
  \item the model may be placed in \textbf{Restricted} status;
  \item capital allocation may be reduced or suspended; and
  \item a formal review must be initiated.
\end{itemize}

\subsubsection{Documentation}

All breaches, responses, and outcomes must be documented in the Model Registry.

\subsection{Performance Monitoring and Interpretation}

Realized performance is monitored as one input into governance, but is not sufficient on its own to determine model validity.

Performance deviations must be interpreted in conjunction with:

\begin{itemize}
  \item changes in behavior metrics;
  \item assumption validity; and
  \item market or structural conditions.
\end{itemize}

Unexpected performance, positive or negative, triggers review when inconsistent with validated expectations.

\subsection{Periodic Review Requirements}

All deployable models must undergo periodic review at intervals no less frequent than quarterly.

Periodic reviews assess:

\begin{itemize}
  \item continued alignment with the declared Operating Context;
  \item relevance of assumptions and constraints;
  \item adequacy of monitoring metrics and thresholds; and
  \item appropriateness of current capital allocation.
\end{itemize}

Reviews must be documented and retained.

\subsection{Authority to Restrict or Suspend Models}

Risk management and designated governance functions are authorized to:

\begin{itemize}
  \item restrict model usage;
  \item reduce capital exposure; or
  \item suspend deployment
\end{itemize}

when monitoring indicates elevated risk, regardless of recent performance.

Such actions are preventative and do not require evidence of loss.

\subsection{Learning and Feedback}

Monitoring results should inform:

\begin{itemize}
  \item future validation standards;
  \item updates to Operating Context declarations; and
  \item improvements in governance controls.
\end{itemize}

Failures and near-misses are treated as learning inputs and are reviewed without attribution of fault.

\subsection{Governing Principle}

\begin{quote}
\textbf{A model remains approved only so long as its behavior remains consistent with the conditions under which it was approved.}
\end{quote}
